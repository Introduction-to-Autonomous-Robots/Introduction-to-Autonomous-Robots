% \chapter{Linear Algebra}
% Linear algebra concerns vector spaces and linear mappings between them. It is central to robotics as it allows describing positions and speeds of the robot within the world as well as moving parts connected to it. 

\chapter{线性代数}

线性代数涉及向量空间和它们之间的线性映射。它是机器人的核心,因为它可以描述世界范围内的机器人的位置和速度以及与之相连的运动部件。

% \section{Dot product}
% The dot product (or scalar product)\index{Dot product}\index{Scalar product} is the sum of the products of the individual entries of two vectors. Let $hat{a}=(a_1,\ldots,a_n)$ and $\hat{b}=(b_1,\ldots,b_n)$ be two vectors. Then, their dot product $\hat{a}\cdot\hat{b}$ is given by

\section{点积}

点积(Dot product)(或标量积(Scalar product))\index{点积(Dot product)}\index{标量积(Scalar product)}是两个向量的各项的乘积之和。令$\hat{a}=(a_1,\ldots,a_n)$和$\hat{b}=(b_1,\ldots,b_n)$是两个向量。那么,它们的点积$\hat{a}\cdot\hat{b}$为

\begin{equation}
\hat{a}\cdot\hat{b}=\sum_{i}^n=a_ib_i
\end{equation}

% The dot product therefore takes two sequences of numbers and returns a single scalar. 

% In robotics, the dot product is mostly relevant due to its geometric interpretation:

因此,点积计算两个数字序列,然后返回一个标量。

在机器人中,由于其几何解释,点积是最相关的概念:

\begin{equation}
\hat{a}\cdot\hat{b}=\|\hat{a}\|\|\hat{b}\|\cos\theta
\end{equation}

% with $\theta$ the angle between vectors $\hat{a}$ and $\hat{b}$. 

% If $\hat{a}$ and $\hat{b}$ are orthogonal, it follows $\hat{a}\cdot\hat{b}=0$. If $\hat{a}$ and $\hat{b}$ are parallel, it follows $\hat{a}\cdot\hat{b}=\|\hat{a}\|\|\hat{b}\|$.  

其中,$\theta$是向量$\hat{a}$和$\hat{b}$之间的角度。

如果$\hat{a}$和$\hat{b}$正交,则$\hat{a}\cdot\hat{b}=0$。如果$\hat{a}$和$\hat{b}$平行,则$\hat{a}\cdot\hat{b}=\|\hat{a}\|\|\hat{b}\|$。

% \section{Cross product}
% The cross product $\hat{a} \times \hat{b}$ of two vectors is defined as a vector $\hat{c}$ that is perpendicular to both $\hat{a}$ and $\hat{b}$. Its direction is given by the right-hand rule and its magnitude is equal to the area of the parallelogram that the vectors span.  

% Let Let $\hat{a}=(a_1,a_2,a_3)^T$ and $\hat{b}=(b_1,a_2,a_3)$ be two vectors in $\mathrm{R}^3$. Then, there cross product $\hat{a}\times\hat{b}$ is given by

\section{叉积}
两个向量的叉积(Cross product)$\hat{a}\times\hat{b}$定义为一个与$\hat{a}$和$\hat{b}$垂直的向量$\hat{c}$。其方向由右手法则给出,其大小等于向量跨越\todo{span}的平行四边形的面积。

令$\hat{a}=(a_1,a_2,a_3)^T$和$\hat{b}=(b_1,a_2,a_3)$是$\mathrm{R}^3$中的两个向量。然后,它们的叉积$\hat{a}\times\hat{b}$为

\begin{equation}
\hat{a}\times\hat{b}=\left(
\begin{array}{l}
a_2b_3-a_3b_2\\
a_3b_1-a_1b_3\\
a_1b_2-a_2b_1
\end{array}
\right)
\end{equation}


% \section{Matrix product}
% Given an $n \times m$ matrix $\mathbf{A}$ and a $m\times p$ matrix $\mathbf{B}$, the matrix product $\mathbf{AB}$ is defined by

\section{矩阵乘法}

给定一个$n\times m$矩阵$\mathbf{A}$和一个$m\times p$矩阵$\mathbf{B}$,矩阵乘积$\mathbf{AB}$为

\begin{equation}
(\mathbf{AB})_{ij}=\sum_{k=1}^mA_{ik}B_{kj}
\end{equation}

% where the index $ij$ indicates the i-th row and j-th column entry of the resulting $n\times p $ matrix. Each entry therefore consists of the scalar product of the i-th row of $\mathbf{A}$ with the j-th column of $\mathbf{B}$. 

其中索引$ij$表示生成的$n\times p$矩阵的第i行和第j列的项。因此,每项等于$\mathbf{A}$的第i行与$\mathbf{B}$的第j列的点积。

% Note that for this two work, the right hand matrix (here $\mathbf{B}$) has to have as many columns as the left hand matrix (here $\mathbf{A}$) has rows. Therefore, the operation is not commutative, i.e., $\mathbf{AB}\neq\mathbf{BA}$.

% For example, multiplying a 3x3 matrix with a 3x1 matrix (a vector), works as follows:
% Let

请注意,对于这两个工作,右边的矩阵(这里$\mathbf{B}$)的列数必须与左边矩阵(这里$\mathbf{A}$)的行数相同。因此,矩阵乘法没有交换律,即$\mathbf{AB}\neq\mathbf{BA}$。

例如,将$3\time 3$矩阵与$3\times 1$矩阵(向量)相乘,如下:

令

\begin{equation}\nonumber
\mathbf{A} = \begin{pmatrix}
a & b & c \\
p & q & r \\
u & v & w
\end{pmatrix} \qquad \mathbf{B} = \begin{pmatrix}
x \\
y \\
z
\end{pmatrix}.
\end{equation}

% Then their matrix product is:
那么他们的矩阵乘法为:

\begin{equation}\nonumber
\mathbf{AB} = \begin{pmatrix}
a & b & c \\
p & q & r \\
u & v & w
\end{pmatrix} \begin{pmatrix}
x \\
y \\
z
\end{pmatrix} =\begin{pmatrix}
ax + by + cz \\
px + qy + rz \\
ux + vy + wz
\end{pmatrix}
\end{equation}

% \section{Matrix inversion}
% Given a matrix $\mathbf{A}$, finding the inverse $\mathbf{B}=\mathbf{A}^{-1}$ involves solving the system of equations that satisfies

\section{矩阵的逆}

给定矩阵$\mathbf{A}$,要得到它的逆$\mathbf{B}=\mathbf{A}^{-1}$我们需要求解满足如下条件的方程组

\begin{equation}
\mathbf{AB}=\mathbf{BA}=\mathbf{I}
\end{equation}

% With $\mathbf{I}$ the identity matrix\index{Identify matrix}. (The identity matrix is zero everywhere expect at its diagonal entries, which are one.)

% In the particular case of orthonormal matrices\index{Orthonormal matrix}, which columns are all orthogonal to each other and of length one, the inverse is equivalent to the transpose, i.e.

其中$\mathbf{I}$为单位矩阵(Identify matrix)\index{单位矩阵(Identify matrix)}。(单位矩阵中对角线上的项为一,其它的项为零)。

对正交矩阵(Orthonormal matrix)\index{正交矩阵(Orthonormal matrix)}(列彼此正交,长度为$1$)求逆,相当于转置,即

\begin{equation}
\mathbf{A}^{-1}=\mathbf{A}^T
\end{equation}

% In case a matrix is not quadratic, we can calculate the pseudo-inverse\index{Pseudo-inverse}, which is defined by

在矩阵不是二次的情况下,我们可以计算伪逆(Pseudo-inverse)\index{伪逆(Pseudo-inverse)}

\begin{equation}
\mathbf{A}^+=\mathbf{A}^T(\mathbf{AA}^T)^{-1}.
\end{equation}