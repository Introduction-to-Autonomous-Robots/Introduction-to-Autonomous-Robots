% \chapter{RGB-D SLAM}\label{rgbd-slam}

% Range sensors have emerged as one of the most effective sensors to make robots autonomous. Unlike vision, range data makes the construction of a 3D model of the robot's environment straightforward and the Velodyne sensor, that combines 64 scanning lasers into one package, was key in mastering the DARPA Grand Challenge.  3D range data has become even more important in robotics with the advent of cheap (priced at a tenth than the cheapest 2D laser scanner) RGB-D (color image plus depth) cameras. Point cloud data allows fitting of lines using RANSAC, which can serve as features in EKF-based localization, but can also be used for improving odometry, loop-closure detection, and mapping. The goals of this chapter are

\chapter{RGB-D SLAM} 
\label{rgbd-slam}

范围传感器已成为使机器人自主的最有效的传感器之一。与视觉不同,范围数据使得机器人环境的3D模型的构建简单直观,Velodyne传感器将64个扫描激光器组合成一个封装,是掌握DARPA Grand Challenge的关键。随着近来价格便宜(价格比最便宜的二维激光扫描仪)RGB-D(彩色图像加深度)相机出现,3D范围数据变得更加重要。点云数据允许使用RANSAC拟合线,其可以作为基于EKF的定位的特征,但也可用于改进测距,闭环检测和映射。 本章的目标是

\begin{itemize}
% \item introduce the Iterative Closest Point (ICP) algorithm
% \item show how ICP can be improved by providing initial guesses via RANSAC
% \item show how SIFT features can be used to improve point selection and loop-closure in ICP to achieve RGB-D mapping

\item 介绍迭代最近点(ICP)算法
\item 显示如何通过RANSAC提供初始猜测来改进ICP
\item 显示SIFT功能如何可以用于改进ICP中的点选择和闭环,以实现RGB-D映射
\end{itemize}

% \section{Converting range data into point cloud data}
% Point cloud data can be thought of a 3D matrix that maps a certain volume in 3D space. Each cell in this matrix, also known as \emph{Voxel}\index{Voxel}, corresponds to whether there is an obstacle in this volume or not. Different intensity values could correspond to the uncertainty with which this space is to be known to be an obstacle. An efficient method to turn range information into such an uncertainty 3D map is described in \cite{curless96} and became known as \emph{Truncated Surface Distance Function} (TSDF)\index{Truncated Surface Distance Function (TSDF)}\index{TSDF}, commonly referred to as ``Point cloud''\index{Point cloud}.

\section{将范围数据转换为点云数据}
点云数据可以被认为是在3D空间中映射某个体积的3D矩阵。该矩阵中的每个单元格,也称为\emph{Voxel}\index{Voxel},对应于此卷中是否存在障碍。不同的强度值可以对应于该空间被认为是障碍物的不确定性。在\cite{curless96}中描述了将范围信息转换为这种不确定性3D映射的有效方法,并被称为\emph{截断表面距离函数}(TSDF)\index{截断表面距离函数(TSDF)}\index{TSDF},通常称为“点云”\index{点云(Point cloud)}。

% \section{The Iterative Closest Point (ICP) algorithm}
% The \emph{Iterative Closest Point} (ICP)\index{Iterative Closest Point}\index{ICP} algorithm was presented in the early 1990s for registration of 3D range data to CAD models of objects. A more in-depth overview of what is described here is given in \cite{rusinkiewicz01}. The key problem can be reduced to find the best transformation that minimizes the distance between two point clouds. This is the case when matching snapshots from a range sensor or matching a range image with a point cloud sampled from a 3D representation of an object.

% In robotics, ICP found an application to match scans from 2D laser range scanners. For example, the transformation that minimizes the error between two consecutive snapshots of the environment is proportional to the motion of the robot. This is a hard problem as it is unclear, which points in the two consecutive snapshots are ``pairs", which of the points are outliers (due to noisy sensors), and which points need to be discarded as not all points overlap in both snapshots. Stitching a series of snapshots together theoretically allows to create a 2D map of the environment. This is difficult, however, as the error between every snapshots --- similar to odometry --- accumulates.   The ICP algorithm also works in 3D where it allows to infer the change in 6D pose of a camera and creation of 3D maps. In addition, ICP has proven useful for identifying objects from a database of 3D objects.

% Before providing a solution to the mapping problem, we will focus on the ICP algorithm to match 2 consecutive frames. Variants of the ICP algorithm can be broken down into 6 consecutive steps:

\section {迭代最近点(ICP)算法}
在20世纪90年代初提出了\emph{迭代最近点}(ICP)\index{Iterative Closest Point} \index {ICP}算法,用于将3D范围数据注册到对象的CAD模型。在\cite{rusinkiewicz01}中给出了更详细的概述。可以减少关键问题,找到最小化两点云之间距离的最佳转换。当从范围传感器匹配快照或者与从对象的3D表示中采样的点云匹配的范围图像时,就是这种情况。

在机器人技术方面,ICP发现了一个应用来匹配来自2D激光范围扫描仪的扫描。例如,最小化环境的两个连续快照之间的错误的转换与机器人的运动成比例。这是一个困难的问题,因为它不清楚,连续两个快照中的哪些点是“对”,哪些点是离群值(由于噪声传感器),哪些点需要丢弃,因为并不是所有点都重叠快照将理论上的一系列快照拼接在一起可以创建一个环境的二维图,但这是困难的,因为每个快照之间的误差 - 类似于odometry ---累积,ICP算法也可以在3D中它可以推断出摄像机的6D姿态的变化和3D地图的创建。此外,ICP已被证明对于从3D对象的数据库中识别对象很有用。

在提供映射问题的解决方案之前,我们将重点关注ICP算法来匹配2个连续帧。 ICP算法的变体可以分为6个步骤:

\begin{enumerate}
% \item Selection of points in one or both meshes or point clouds.
% \item Matching/Pairing these points to samples in the other point cloud/mesh.
% \item Weighting the corresponding pairs.
% \item Rejecting certain pairs.
% \item Assigning an error metric based on the point pairs.
% \item Minimizing the error metric.
% \item Point Selection

\item 选择一个或两个网格或点云中的点。
\item 将这些点与其他点云/网格中的样本进行匹配/配对。
\item 加权对应的对。
\item 拒绝某些对。
\item 根据点对分配错误度量。
\item 最小化错误度量。
\item 点选择
\end{enumerate}

% Depending on the number of points generated by the range sensor, it might make sense to use only a few selected points to calculate the optimal transformation between two point clouds, and then test this transformation on all points. Depending on the source of the data, it also turns out that some points are more suitable than others as it is easier to identify matches for them. This is the case for RGB-D data, where SIFT features have been used successfully. This is also the case for planar objects with grooves, where sampling should ensure that angles of normal vectors of sampling points are broadly distributed. Which method to use is therefore strongly dependent on the kind of data being used and should be considered for each specific problem.

根据距离传感器产生的点数,使用只有几个选定的点来计算两个点云之间的最佳变换可能是有意义的,然后在所有点上测试此转换。根据数据的来源,它也证明一些点比其他点更合适,因为它们更容易识别它们的匹配。这是RGB-D数据的情况,SIFT功能已成功使用。对于具有槽的平面物体也是如此,其中采样应确保采样点的法向矢量的角度广泛分布。因此,使用哪种方法很大程度上取决于所使用的数据类型,应针对每个具体问题进行考虑。

% \subsection{Matching Points}
% The key step in ICP is to match one point to its corresponding point.  For example, a laser scanner hits a certain point at a wall with its 67th ray. After the scanner has been moved by 10 cm, the closest hit on the wall to this point might have been by the 3rth ray of the laser. Here, it is actually very unlikely that the laser hits the exact same point on the wall twice, therefore introducing a non-zero error even for optimal pairing. Prominent methods are to find the closest point in the other point cloud or to find the intersection of the source points normal with the destination surface (for matching point clouds to meshes). More recently, SIFT has allowed to match points based on their visual appearance. Similarly to sorting through SIFT features, finding the closest matching point can be accelerated by representing the point cloud in a k-d tree.

\subsection{Matching Points}
ICP的关键步骤是将一点与其对应点相匹配。例如,激光扫描仪用第67号射线在墙上的某一点击中。在扫描仪移动了10厘米后,墙上最接近的这一点可能是通过激光的第3条光线。在这里,激光器实际上不太可能在墙上两次完全相同的点,因此即使是最佳配对也引入非零误差。突出的方法是找到另一个点云中的最近点,或者找到源点与目标表面正交(用于匹配点云到网格)的交点。最近,SIFT已经允许基于他们的视觉外观来匹配点。类似于通过SIFT特征的排序,通过表示k-d树中的点云可以加快找到最接近的匹配点。

% \subsection{Weighting of Pairs}
% As some pairs are better matches than others, weighting them in some smart way might drastically improve the quality of the resulting transformation. One approach is to give more weight to points that have smaller distances from each other. Another approach is to take into account the color of the point (in RGB-D images) or use the distance of their SIFT features (weighting pairs with low distances higher than pairs with high distances). Finally, expected noise can be used to weight pairings. For example, the estimates made by a laser scanner are much more faithful when taken orthogonally to a plane than when taken at a steep angle.

\subsection {对的加权}
由于某些配对比其他配对更好,因此以一些聪明的方式对它们进行加权可能会大大提高产生的转换质量。一种方法是对彼此距离较小的点给予更多的权重。另一种方法是考虑到点(RGB-D图像)的颜色,或者使用它们的SIFT特征的距离(具有高距离的距离高的加权对)。最后,预期的噪声可以用于加权配对。例如,当与陡峭角度拍摄时相比,激光扫描仪的估计值与正交平面相比更加忠实。

% \subsection{Rejecting of Pairs}
% A key problem in ICP are outliers either from sensor noise or simply from incomplete overlap between two consecutive range images. A prime approach in dealing with this problem is to reject pairings of which one of the points lies on a boundary of the point cloud as these points are likely to match with points in non-overlapping regions. As a function of the underlying data, it might also make sense to reject pairings with too high of a distance. This is a threshold-based equivalent to distance-based weighting as described above.

\subsection{对对方拒绝}
ICP的一个关键问题是来自传感器噪声或简单地来自两个连续范围图像之间不完全重叠的异常值。处理这个问题的一个主要方法是拒绝哪一点位于点云边界上的配对,因为这些点可能与非重叠区域中的点相匹配。作为底层数据的功能,也可能会忽略距离太远的配对。这是基于阈值的等价于基于距离的权重,如上所述。

% \subsection{Error Metric and Minimization Algorithm}
% After points have been selected and matched, pairs have been weighted and rejected, the match between two point clouds needs to be expressed by a suitable error metric, which needs then to be minimized. A straightforward approach is to consider the sum of squared distances between each pair. This formulation can often be solved analytically. Let

\subsection{错误度量和最小化算法}
在选择和匹配点之后,对已经被加权和拒绝,两个点云之间的匹配需要由适当的误差度量来表示,然后需要将其最小化。一个简单的方法是考虑每对之间的平方距离之和。这个表达式通常可以通过分析来解决。 令

\begin{eqnarray}
A=\{a_1,\ldots,a_n\}\\
B=\{b_1,\dots,b_n\}
\end{eqnarray}

% be point clouds in $ \mathbb{R}^n$. The goal is now to find a vector $ t \in \mathbb{R}^n$ so that an error function $ \phi(A+t,B)$ is minimized. In 6D (translation and rotation), an equivalent notation can be found for a transformation (see forward kinematics). An error function for the squared distance is then given by

在$\mathbb{R}^n$中的点云。目标是在中找到一个向量$t\in\mathbb{R}^n$,以使最小化错误函数$\phi(A+t,B)$。在6D(平移和旋转)中,可以找到一个转换的等效符号(见正运动学)。然后给出平方距离的误差函数

\begin{equation}
\phi(A+t,B)=\frac{1}{n}\sum_{a \in A}\|a+t-N_B(a+t)\|^2
\end{equation}

% Here $ N_B(a+t)$ is a function that provides the nearest neighbor of $ a$ translated by $ b$ in $ B$.  A key problem now is that the actual value of $t$ affects the outcome of the pairing. What might look like a good match initially often turns out not be the final pairing. A simple numerical approach to this problem is to find $ t$ iteratively.

% Initially $t=0$ and nearest neighbors/pairings are established. We can now calculate a $ \delta t$ that optimizes the least-square problem based on this matching using any solver available for the optimization problem (for a least-square solution $ \delta t$ can be obtained analytically by solving for the minimum of the polynomial by setting its derivative to zero). We can then shift all points in $ A$ by $ \delta t$ and start over. That is, we calculate new pairings and derive a new $ \delta t$.  We can continue to do this, until the cost function reaches a local minimum.

% Instead of formulating the cost function as a ``point-to-point'' distance, a ``point-to-plane'' has become popular. Here, the cost function consist of the sum of squared distances from each source point to the plane that contains the destination point and is oriented perpendicular to the destination normal. This makes particularly sense when matching a point cloud to a mesh/CAD model of an object. In this case there are no analytical solutions to finding the optimal transformation, but any optimization method such a Levenberg-Marquardt can be used.

这里$N_B(a+t)$是一个函数,它提供$B$中$a$的最近邻居$b$。现在的关键问题是$t$的实际值影响配对的结果。最初似乎是一个很好的搭配,往往不是最终配对。这个问题的一个简单的数值方法是迭代地找到$t$。

最初$t=0$和最近邻/配对建立。我们现在可以计算一个$\delta t$,它可以使用可用于优化问题的任何解算器,根据这种匹配优化最小二乘问题(对于最小二乘解决方案$\delta t$可以通过解析求最小化来获得通过将其导数设置为零)。然后我们可以将$A$中的所有点转移$\delta t$并重新开始。也就是说,我们计算新的配对并导出一个新的$\delta t$。我们可以继续这样做,直到成本函数达到局部最小值。

成本函数作为“点到点”的距离,而不是“点到平面”,已经变得流行起来了。这里,成本函数包括从每个源点到包含目的地点的平面的平方距离之和并且垂直于目的地法线。当将点云匹配到对象的网格/CAD模型时,这是特别有意义的。在这种情况下,没有找到最佳变换的分析解决方案,但是可以使用诸如Levenberg-Marquardt之类的任何优化方法。

% \section{RGB-D Mapping}
% The ICP algorithm can be used to stitch consecutive range images together to create a 3D map of the environment \cite{henry2010rgb}. Together with RGB information, it is possible to create complete 3D walk throughs of an environment. An example of such a walk through using the method described in \cite{whelan2013robust} is shown in Figure \ref{fig:kintinous}.

% A problem with ICP is that errors in each transformation propagate making maps created using this method as odd as maps created by simple odometry. Here, the SLAM algorithm can be used to correct previous errors once a loop closure is detected.

\section{RGB-D Mapping}
ICP算法可用于将连续范围图像拼接在一起,以创建环境的3D地图\cite{henry2010rgb}。与RGB信息一起,可以创建完整的3D漫步环境。通过使用\cite{whelan2013robust}中描述的方法进行的这种漫步的例子如图\ref{fig:kintinous}所示。

ICP的一个问题是,每个变换中的错误传播,使用这种方法创建的地图与由简单测距法创建的地图一样奇怪。一旦检测到环路闭合,SLAM算法就可用于校正以前的错误。

\begin{figure}
\centering
\includegraphics[width=\textwidth]{figs/kintinous}
% \caption{Fused point cloud data from a walk trough of an office environment using ``Kintinious''. Picture courtesy of John Leonard.}
\caption{使用“Kintinious”的办公环境的散步槽的融合点云数据。 图片由约翰·伦纳德提供。}
\label{fig:kintinous}
\end{figure}

% The intuition behind SLAM is to consider each transformation between consecutive snapshots as a spring with variable stiffness. Whenever the robot returns to a previously seen location, i.e., a loop-closure has been determined, additional constraints are introduced and the collection of snapshots connected by springs become a mesh. Everytime the robot then re-observes a transformation between any of the snapshots, it can ``stiffen'' the spring connecting the two. As all of the snapshots are connected, this new constraints propagates through the network and literally pull each snapshots in place.

% RGB-D Mapping uses a variant of ICP that is enhanced by SIFT features for point selection and matching. Maps are build incrementally. SIFT features, and their spatial relationship, are used for detecting loop closures. Once a loop closure is detected, an additional constraint is added to the pose graph and a SLAM-like optimization algorithm corrects the pose of all previous observations.

% As ICP only works when both point clouds are already closely aligned, which might not be the case for a fast moving robot with a relatively noisy sensor (the XBox Kinect has an error of 3cm for a few meters of range vs. millimeters in laser range scanners), RGB-D Mapping uses RANSAC to find an initial transformation. Here, RANSAC works as for line fitting: it keeps guessing possible transformations for 3 pairs of SIFT feature points and then counts the number of inliers when matching the two point clouds, one of which being transformed using the random guess.

SLAM背后的直觉是将连续快照之间的每个变换视为具有可变刚度的弹簧。每当机器人返回到先前看到的位置时,即已经确定了闭环,则引入额外的约束,并且通过弹簧连接的快照的集合变成网格。每当机器人重新观察任何快照之间的转换时,它可以“连接”连接两个快照的弹簧。由于所有的快照都是连接的,所以新的约束通过网络传播,并逐字地将每个快照拉到位。

RGB-D映射使用由SIFT功能增强的ICP的变体,用于点选择和匹配。地图逐渐建立。SIFT特征及其空间关系用于检测环路闭合。一旦检测到环路闭合,一个额外的约束被添加到姿态图,并且SLAM类优化算法校正所有先前观察的姿态。

由于ICP仅在两个点云已经紧密对齐的情况下才起作用,对于具有相对噪声传感器的快速移动机器人而言,这可能不是这种情况(XBox Kinect在几米范围内的误差为3厘米,而激光范围内为毫米扫描仪),RGB-D映射使用RANSAC来找到初始变换。这里,RANSAC适用于线拟合:它持续猜测3对SIFT特征点的可能变换,然后计算匹配两点云时的内部数量,其中一个使用随机猜测进行变换。