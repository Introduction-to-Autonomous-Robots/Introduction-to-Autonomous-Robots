%% Based on a TeXnicCenter-Template by Tino Weinkauf.
%%%%%%%%%%%%%%%%%%%%%%%%%%%%%%%%%%%%%%%%%%%%%%%%%%%%%%%%%%%%%

%%%%%%%%%%%%%%%%%%%%%%%%%%%%%%%%%%%%%%%%%%%%%%%%%%%%%%%%%%%%%
%% HEADER
%%%%%%%%%%%%%%%%%%%%%%%%%%%%%%%%%%%%%%%%%%%%%%%%%%%%%%%%%%%%%
\documentclass[letter,twoside,11pt]{article}
% Alternative Options:
%	Paper Size: a4paper / a5paper / b5paper / letterpaper / legalpaper / executivepaper
% Duplex: oneside / twoside
% Base Font Size: 10pt / 11pt / 12pt


%% Language %%%%%%%%%%%%%%%%%%%%%%%%%%%%%%%%%%%%%%%%%%%%%%%%%
\usepackage[USenglish]{babel} %francais, polish, spanish, ...
\usepackage[T1]{fontenc}
\usepackage[ansinew]{inputenc}

\usepackage{lmodern} %Type1-font for non-english texts and characters


%% Packages for Graphics & Figures %%%%%%%%%%%%%%%%%%%%%%%%%%
\usepackage{graphicx} %%For loading graphic files
%\usepackage{subfig} %%Subfigures inside a figure
%\usepackage{pst-all} %%PSTricks - not useable with pdfLaTeX

%% Please note:
%% Images can be included using \includegraphics{Dateiname}
%% resp. using the dialog in the Insert menu.
%% 
%% The mode "LaTeX => PDF" allows the following formats:
%%   .jpg  .png  .pdf  .mps
%% 
%% The modes "LaTeX => DVI", "LaTeX => PS" und "LaTeX => PS => PDF"
%% allow the following formats:
%%   .eps  .ps  .bmp  .pict  .pntg


%% Math Packages %%%%%%%%%%%%%%%%%%%%%%%%%%%%%%%%%%%%%%%%%%%%
\usepackage{amsmath}
\usepackage{amsthm}
\usepackage{amsfonts}


%% Line Spacing %%%%%%%%%%%%%%%%%%%%%%%%%%%%%%%%%%%%%%%%%%%%%
%\usepackage{setspace}
%\singlespacing        %% 1-spacing (default)
%\onehalfspacing       %% 1,5-spacing
%\doublespacing        %% 2-spacing


%% Other Packages %%%%%%%%%%%%%%%%%%%%%%%%%%%%%%%%%%%%%%%%%%%
%\usepackage{a4wide} %%Smaller margins = more text per page.
%\usepackage{fancyhdr} %%Fancy headings
%\usepackage{longtable} %%For tables, that exceed one page


%%%%%%%%%%%%%%%%%%%%%%%%%%%%%%%%%%%%%%%%%%%%%%%%%%%%%%%%%%%%%
%% Remarks
%%%%%%%%%%%%%%%%%%%%%%%%%%%%%%%%%%%%%%%%%%%%%%%%%%%%%%%%%%%%%
%
% TODO:
% 1. Edit the used packages and their options (see above).
% 2. If you want, add a BibTeX-File to the project
%    (e.g., 'literature.bib').
% 3. Happy TeXing!
%
%%%%%%%%%%%%%%%%%%%%%%%%%%%%%%%%%%%%%%%%%%%%%%%%%%%%%%%%%%%%%

%%%%%%%%%%%%%%%%%%%%%%%%%%%%%%%%%%%%%%%%%%%%%%%%%%%%%%%%%%%%%
%% Options / Modifications
%%%%%%%%%%%%%%%%%%%%%%%%%%%%%%%%%%%%%%%%%%%%%%%%%%%%%%%%%%%%%

%\input{options} %You need a file 'options.tex' for this
%% ==> TeXnicCenter supplies some possible option files
%% ==> with its templates (File | New from Template...).



%%%%%%%%%%%%%%%%%%%%%%%%%%%%%%%%%%%%%%%%%%%%%%%%%%%%%%%%%%%%%
%% DOCUMENT
%%%%%%%%%%%%%%%%%%%%%%%%%%%%%%%%%%%%%%%%%%%%%%%%%%%%%%%%%%%%%
\begin{document}
\vspace{-50px}
\title{Homework 2}
\author{Introduction to Robotics\\
ncorrell@colorado.edu}
%\date{} %%If commented, the current date is used.
\maketitle

\begin{enumerate}
\item How does the computational complexity of Dijkstra's algorithm change when moving from 2D to 3D search spaces? (1pt)
\item A* uses a ``heuristic'' to bias the search in the expected direction of the goal. Why can it only use a heuristic, not the actual length? (1pt)
\item Assuming points are sampled uniformly at random in a randomized planning algorithm. Calculate the limiting behaviour of the following ratio   (number of points in tree)/(number of points sampled) as the number of points sampled goes to infinity. Assume the total area $A_{total}$ and the area of free space $A_{free}$ within are known. (1pt) 

\item Assuming a kd-tree is used as a nearest-neighbour data structure, and points are sample uniformly at random, calculate the  run-time of inserting a point into a tree of size $N$. Use ``big-Oh'' notation, e.g. $\mathcal{O}(N)$. (1pt)

\item Why does the bandwidth of a Ultra-sound based distance sensor decreases significantly when increasing its dynamic range, but that of a laser range scanner does not for typical operation? (1pt)

\item You are designing an autonomous electric car to transport goods on campus. As you are worried about cost, you are thinking about whether to use a laser scanner or an ultra-sound sensor for detecting obstacles. As you drive rather slow, you are required to sense up to 15 meters. The laser scanner you are considering can sense up to this range and has a bandwidth of 10Hz. Assume 300m/s for the speed of sound in the following.
\begin{enumerate}
\item Calculate the time it takes until you hear back from the US sensor when detecting an obstacle 15m away. Assume that the robot is not moving at this point. (1pt)
\item Calculate the time it takes until you hear back from the laser scanner. Hint: you don't need the speed of light for this, the answer is in the specs above. (1pt)     
\item Assume now that you are moving toward the obstacle. Which sensor will give you a measurement that is closer to your real distance at the time of reading and why? (1pt) 
\end{enumerate}
\end{enumerate}

%% Title Page %%%%%%%%%%%%%%%%%%%%%%%%%%%%%%%%%%%%%%%%%%%%%%%
%% ==> Write your text here or include other files.

%% The simple version:


%% The nice version:
%\input{titlepage} %%You need a file 'titlepage.tex' for this.
%% ==> TeXnicCenter supplies a possible titlepage file
%% ==> with its templates (File | New from Template...).
\end{document}

